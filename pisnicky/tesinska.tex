\beginsong{Těšínská}[by={Jaromír Nohavica}]

\beginverse
\[Ami]Kdybych se narodil \[Dmi]před sto léty
\[F,] \[E7]v tomhle \[Ami]městě \[Dmi, F, E7, Ami]
\[Ami]u Larichů na zahradě \[Dmi]trhal bych květy
\[F,] \[E7]své ne\[Ami]věstě.\[ Dmi, F, E7, Ami]
\[C]Moje nevěsta by \[Dmi]byla dcera ševcova
\[F]z domu Kamińskich \[C]odněkud ze Lvova
kochał bym ją i \[Dmi]pieśćił \[F]chy\[E7]ba lat \[Ami]dwieśćie.
\endverse
\beginverse
Bydleli bychom na Sachsenbergu v domě u žida Kohna.
Nejhezčí ze všech těšínských šperků byla by ona.
Mluvila by polsky a trochu česky,
pár slov německy a smála by se hezky.
Jednou za sto let zázrak se koná, zázrak se koná.
\endverse
\beginverse
Kdybych se narodil před sto léty byl bych vazačem knih.
U Prohazků dělal bych od pěti do pěti a sedm zlatek za to bral bych.
Měl bych krásnou ženu a tři děti,
zdraví bych měl a bylo by mi kolem třiceti,
celý dlouhý život před sebou celé krásné dvacáté století.
\endverse
\beginverse
Kdybych se narodil před sto léty v jinačí době
u Larichů na zahradě trhal bych květy má lásko tobě.
Tramvaj by jezdila přes řeku nahoru,
slunce by zvedalo hraniční závoru
a z oken voněl by sváteční oběd.
\endverse
\beginverse
Večer by zněla od Mojzese melodie dávnověká,
bylo by léto tisíc devět set deset za domem by tekla řeka.
Vidím to jako dnes šťastného sebe,
ženu a děti a těšínské nebe.
Jěště,že člověk nikdy neví co ho čeká.
\endverse
\beginverse*
na na na na... 
\endverse
\endsong
